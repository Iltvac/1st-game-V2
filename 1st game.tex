\documentclass{article}
\usepackage {cmap} 
\usepackage [T2A] {fontenc}
\usepackage [utf8]{inputenc}
\usepackage [english,russian]{babel}
\usepackage{graphicx}
\DeclareGraphicsExtensions{.pdf,.png,.jpg}

\title{1st Game}
\author{Дерюгин Никита }
\date{Декабрь 2023}

\begin{document}
	
	\maketitle
	Для начала стоит сказать,что идея игры,возможно,не является чем-то редким и уникальным,а даже наоборот,довольно популярной.Материалы(ассеты) для создания игры также были взяты из открытых источников.
	\section{Вступление}
	После получения задания,которое говорит нам создать свою первую 2D игру,мне предстояло определиться только с тем,какую игру я хочу,ну или смогу(т.к опыт в программировании у меня небольшой,а в разработке чего-то вообще отсутствует) создать.Над выбором движка,на котором всё это будет происходить я не колебался,так как была рекомендация попробоваться Godot Engine.
	Стоит вернуться к выбору идеи своей игры,я не стал далеко ходить и решил реализовать игру по такому смыслу:есть персонаж,который умеет бегать и прыгать,а также есть поверхность,на которой собственно и будет бегать/прыгать персонаж.
	\section {Начало разработки}
	Разработка игры,в моем случае,началась с импорта всех необоходимых мне ассетов.После была создана сцена,где персонаж обзавелся коллизией,а также первая платформа(для нее также была добавлена коллизия),где будет стоять персонаж.После этого уже можно добавлять анимации для движения персонажа и обучать его ходить и прыгать через скрипт.
	\section{Разработка}
	После того,как наш персонаж научился ходить,и мы учли все тонкости его управление,т.е чтобы анимация персонажа соответствовала его действиям,нажатия клавиш были корректны,константы гравитации,скорости движения и скорости прыжка также должны быть настроены корректно,иначе,допустим при нажатии на кнопку прыжка персонаж будет улетать в неизвестност.Как только всё это выполнено,плавно можно перейти к созданю самого уровня.
	Ниже представлен почти что весь код моей игры.
		\begin{figure}
		\centering
		\includegraphics[width=0.5\linewidth]{Снимок экрана 2023-12-24 185443.png}
		
	\end{figure}
	Стоит вернуть
	\section{Создание уровня}
	Уровень будет представлять собой n-ое количество платформ,по которым персонаж может перемещаться тем или иным способом(бегать или же прыгать).Создаем нод TileMap,в него загружаем необходимые текстуры.Дальше уже применяя фантазию создаем уровень,но не забываем добавить для текстур коллизию,иначе персонаж просто будет проваливаться.У каждого уровня должно быть какое-то логическое завершение,в моем случае не будет каких-то монеток или чего-то подобного,собирая которые мы могли бы перемещаться на следующий уровень,вместо этого я решил просто добавить некую конечную точку,которая обозначена факелами,достигая ее мы проваливаемся,после чего у нас появляется панель,об окончании игры,это и будет концом нашего уровня.
	\section {Окно об окончании игры}
	Раз уж я затронул окно об окончании игры,расскажу про его создание.В целом,это не является чем-то сложным,мы добавляем фон,кнопки с текстом.Только для того,чтобы кнопки заработали нужно будет использоваться функцию сигналов,это работает так:при нажатии,допустим,на кнопку "next level" будет происходить некое действие,в нашем случае это переход на следующий уровень.Остальные кнопки можно сделать рабочими по точно такому же принципу,только ожидаемое действие будет изменяться.
		\begin{figure}
		\centering
		\includegraphics[scale=0.2]{окно.png}
		\caption{окно об окончании игры}
		
	\end{figure}
	\section{Заключение}
	В этом небольшом отчете я в кратце постарался рассказать про то,как я создавал свою первую игру.Да,она не является какой-то супер сложной или захватывающей,но при её создании я в первую очередь знакомился с Godot'ом,что для меня,несомненно,является важным и довольно интересным опытом.
\end{document}


